% Options for packages loaded elsewhere
\PassOptionsToPackage{unicode}{hyperref}
\PassOptionsToPackage{hyphens}{url}
\PassOptionsToPackage{dvipsnames,svgnames,x11names}{xcolor}
%
\documentclass[
  letterpaper,
  DIV=11,
  numbers=noendperiod]{scrartcl}

\usepackage{amsmath,amssymb}
\usepackage{lmodern}
\usepackage{iftex}
\ifPDFTeX
  \usepackage[T1]{fontenc}
  \usepackage[utf8]{inputenc}
  \usepackage{textcomp} % provide euro and other symbols
\else % if luatex or xetex
  \usepackage{unicode-math}
  \defaultfontfeatures{Scale=MatchLowercase}
  \defaultfontfeatures[\rmfamily]{Ligatures=TeX,Scale=1}
\fi
% Use upquote if available, for straight quotes in verbatim environments
\IfFileExists{upquote.sty}{\usepackage{upquote}}{}
\IfFileExists{microtype.sty}{% use microtype if available
  \usepackage[]{microtype}
  \UseMicrotypeSet[protrusion]{basicmath} % disable protrusion for tt fonts
}{}
\makeatletter
\@ifundefined{KOMAClassName}{% if non-KOMA class
  \IfFileExists{parskip.sty}{%
    \usepackage{parskip}
  }{% else
    \setlength{\parindent}{0pt}
    \setlength{\parskip}{6pt plus 2pt minus 1pt}}
}{% if KOMA class
  \KOMAoptions{parskip=half}}
\makeatother
\usepackage{xcolor}
\setlength{\emergencystretch}{3em} % prevent overfull lines
\setcounter{secnumdepth}{-\maxdimen} % remove section numbering
% Make \paragraph and \subparagraph free-standing
\ifx\paragraph\undefined\else
  \let\oldparagraph\paragraph
  \renewcommand{\paragraph}[1]{\oldparagraph{#1}\mbox{}}
\fi
\ifx\subparagraph\undefined\else
  \let\oldsubparagraph\subparagraph
  \renewcommand{\subparagraph}[1]{\oldsubparagraph{#1}\mbox{}}
\fi


\providecommand{\tightlist}{%
  \setlength{\itemsep}{0pt}\setlength{\parskip}{0pt}}\usepackage{longtable,booktabs,array}
\usepackage{calc} % for calculating minipage widths
% Correct order of tables after \paragraph or \subparagraph
\usepackage{etoolbox}
\makeatletter
\patchcmd\longtable{\par}{\if@noskipsec\mbox{}\fi\par}{}{}
\makeatother
% Allow footnotes in longtable head/foot
\IfFileExists{footnotehyper.sty}{\usepackage{footnotehyper}}{\usepackage{footnote}}
\makesavenoteenv{longtable}
\usepackage{graphicx}
\makeatletter
\def\maxwidth{\ifdim\Gin@nat@width>\linewidth\linewidth\else\Gin@nat@width\fi}
\def\maxheight{\ifdim\Gin@nat@height>\textheight\textheight\else\Gin@nat@height\fi}
\makeatother
% Scale images if necessary, so that they will not overflow the page
% margins by default, and it is still possible to overwrite the defaults
% using explicit options in \includegraphics[width, height, ...]{}
\setkeys{Gin}{width=\maxwidth,height=\maxheight,keepaspectratio}
% Set default figure placement to htbp
\makeatletter
\def\fps@figure{htbp}
\makeatother
\newlength{\cslhangindent}
\setlength{\cslhangindent}{1.5em}
\newlength{\csllabelwidth}
\setlength{\csllabelwidth}{3em}
\newlength{\cslentryspacingunit} % times entry-spacing
\setlength{\cslentryspacingunit}{\parskip}
\newenvironment{CSLReferences}[2] % #1 hanging-ident, #2 entry spacing
 {% don't indent paragraphs
  \setlength{\parindent}{0pt}
  % turn on hanging indent if param 1 is 1
  \ifodd #1
  \let\oldpar\par
  \def\par{\hangindent=\cslhangindent\oldpar}
  \fi
  % set entry spacing
  \setlength{\parskip}{#2\cslentryspacingunit}
 }%
 {}
\usepackage{calc}
\newcommand{\CSLBlock}[1]{#1\hfill\break}
\newcommand{\CSLLeftMargin}[1]{\parbox[t]{\csllabelwidth}{#1}}
\newcommand{\CSLRightInline}[1]{\parbox[t]{\linewidth - \csllabelwidth}{#1}\break}
\newcommand{\CSLIndent}[1]{\hspace{\cslhangindent}#1}

\usepackage{booktabs}
\usepackage{longtable}
\usepackage{array}
\usepackage{multirow}
\usepackage{wrapfig}
\usepackage{float}
\usepackage{colortbl}
\usepackage{pdflscape}
\usepackage{tabu}
\usepackage{threeparttable}
\usepackage{threeparttablex}
\usepackage[normalem]{ulem}
\usepackage{makecell}
\usepackage{xcolor}
\KOMAoption{captions}{tableheading}
\makeatletter
\makeatother
\makeatletter
\makeatother
\makeatletter
\@ifpackageloaded{caption}{}{\usepackage{caption}}
\AtBeginDocument{%
\ifdefined\contentsname
  \renewcommand*\contentsname{Table of contents}
\else
  \newcommand\contentsname{Table of contents}
\fi
\ifdefined\listfigurename
  \renewcommand*\listfigurename{List of Figures}
\else
  \newcommand\listfigurename{List of Figures}
\fi
\ifdefined\listtablename
  \renewcommand*\listtablename{List of Tables}
\else
  \newcommand\listtablename{List of Tables}
\fi
\ifdefined\figurename
  \renewcommand*\figurename{Figure}
\else
  \newcommand\figurename{Figure}
\fi
\ifdefined\tablename
  \renewcommand*\tablename{Table}
\else
  \newcommand\tablename{Table}
\fi
}
\@ifpackageloaded{float}{}{\usepackage{float}}
\floatstyle{ruled}
\@ifundefined{c@chapter}{\newfloat{codelisting}{h}{lop}}{\newfloat{codelisting}{h}{lop}[chapter]}
\floatname{codelisting}{Listing}
\newcommand*\listoflistings{\listof{codelisting}{List of Listings}}
\makeatother
\makeatletter
\@ifpackageloaded{caption}{}{\usepackage{caption}}
\@ifpackageloaded{subcaption}{}{\usepackage{subcaption}}
\makeatother
\makeatletter
\@ifpackageloaded{tcolorbox}{}{\usepackage[many]{tcolorbox}}
\makeatother
\makeatletter
\@ifundefined{shadecolor}{\definecolor{shadecolor}{rgb}{.97, .97, .97}}
\makeatother
\makeatletter
\makeatother
\ifLuaTeX
  \usepackage{selnolig}  % disable illegal ligatures
\fi
\IfFileExists{bookmark.sty}{\usepackage{bookmark}}{\usepackage{hyperref}}
\IfFileExists{xurl.sty}{\usepackage{xurl}}{} % add URL line breaks if available
\urlstyle{same} % disable monospaced font for URLs
\hypersetup{
  pdftitle={Bias in estimated short sprint profiles using timing gates due to the flying start: simulation study and proposed solutions},
  pdfauthor={Mladen Jovanović},
  colorlinks=true,
  linkcolor={blue},
  filecolor={Maroon},
  citecolor={Blue},
  urlcolor={Blue},
  pdfcreator={LaTeX via pandoc}}

\title{Bias in estimated short sprint profiles using timing gates due to
the flying start: simulation study and proposed solutions}
\author{Mladen Jovanović}
\date{9/19/22}

\begin{document}
\maketitle
\begin{abstract}
Short sprints have been modeled using the mono-exponential equation that
involves two parameters: (1) maximum sprinting speed (MSS) and (2)
relative acceleration (TAU), most often performed using the timing
gates. In this study, this model is termed the \emph{No correction}
model. Unfortunately, due to the often utilized flying start, a bias is
introduced when estimating parameters. In this paper, (1) two additional
models are proposed (\emph{Estimated TC} and \emph{Estimated FD}) that
aim to correct this bias, and (2) a theoretical simulation study that
provides model performances in estimating parameters is provided. In
conclusion, both \emph{Estimated TC} and \emph{Estimated FD} models
provided more precise parameter estimates, but surprisingly, the
\emph{No correction} model provided higher sensitivity for specific
parameter changes.
\end{abstract}
\ifdefined\Shaded\renewenvironment{Shaded}{\begin{tcolorbox}[enhanced, boxrule=0pt, frame hidden, sharp corners, borderline west={3pt}{0pt}{shadecolor}, breakable, interior hidden]}{\end{tcolorbox}}\fi

\renewcommand*\contentsname{Table of contents}
{
\hypersetup{linkcolor=}
\setcounter{tocdepth}{3}
\tableofcontents
}
\hypertarget{introduction}{%
\section{Introduction}\label{introduction}}

Sprint speed is one of the most distinctive and admired physical
characteristics in sports. In most team sports (e.g., soccer, field
hockey, handball, etc.), short sprints are defined as maximal sprinting
from a standstill across a distance that does not result in deceleration
at the finish. Peak anaerobic power is reached during the first few
seconds (\textless5 \(s\)) of maximal efforts
(\protect\hyperlink{ref-mangineSpeedForcePower2014}{Mangine et al.
2014})); however, the capacity to attain maximal sprint speed is
athlete- and sport-specific. For instance, track and field sprinters are
trained to achieve maximal speed later in a race (i.e., 50-60 \(m\))
(\protect\hyperlink{ref-ward-smithEnergyConversionStrategies2001}{Ward-Smith
2001}), whereas team sport athletes have sport-specific attributes and
reach maximal speed much earlier (i.e., 30-40 \(m\))
(\protect\hyperlink{ref-brownAssessmentLinearSprinting2004}{Brown et al.
2004}). The evaluation of short sprint performance is frequently
included in a battery of fitness tests for various sports, regardless of
the kinematic differences between athletes.

The use of force plates is regarded as the gold standard for analyzing
the mechanical features of sprinting; nevertheless collecting the
profile of a whole sprint presents practical and cost problems
(\protect\hyperlink{ref-samozinoSimpleMethodMeasuring2016}{Samozino et
al. 2016}; \protect\hyperlink{ref-morinSimpleMethodComputing2019}{Morin
et al. 2019}). Radar and laser technology are frequently utilized
laboratory-grade methods
(\protect\hyperlink{ref-buchheitMechanicalDeterminantsAcceleration2014}{Buchheit
et al. 2014};
\protect\hyperlink{ref-jimenez-reyesRelationshipVerticalHorizontal2018}{Jiménez-Reyes
et al. 2018};
\protect\hyperlink{ref-marcote-pequenoAssociationForceVelocity2019}{Marcote-Pequeño
et al. 2019};
\protect\hyperlink{ref-edwardsSprintAccelerationCharacteristics2020}{Edwards
et al. 2020}) that are typically unavailable to sports practitioners.
Timing gates are unquestionably the most prevalent method available for
evaluating sprint performance. Multiple gates are frequently placed at
different distances to capture split times (e.g., 10, 20, 30, and 40
\(m\)), which can now be incorporated into the method for determining
sprint mechanical properties
(\protect\hyperlink{ref-samozinoSimpleMethodMeasuring2016}{Samozino et
al. 2016}; \protect\hyperlink{ref-morinSimpleMethodComputing2019}{Morin
et al. 2019}). Practitioners can utilize the outcomes to explain
individual differences, quantify the effects of training interventions,
and gain a better knowledge of the limiting variables of performance,
which is an advantage of this method.

\hypertarget{mathematical-model}{%
\subsection{Mathematical model}\label{mathematical-model}}

The mono-exponential Equation~\ref{eq-velocity-time} has been used to
model short sprints. It was first proposed by Furusawa et al.
(\protect\hyperlink{ref-doi:10.1098ux2frspb.1927.0035}{1927}) and made
more popular by Clark et al.
(\protect\hyperlink{ref-clarkNFLCombine40Yard2017}{2017}) and Samozino
et al.
(\protect\hyperlink{ref-samozinoSimpleMethodMeasuring2016}{2016}).
Equation~\ref{eq-velocity-time} is the function for instantaneous
horizontal velocity \(v\) given time \(t\) and two model parameters.

\begin{equation}\protect\hypertarget{eq-velocity-time}{}{
  v(t) = MSS \times (1 - e^{-\frac{t}{TAU}}) 
}\label{eq-velocity-time}\end{equation}

\emph{Maximum sprinting speed} (MSS; expressed in \(ms^{-1}\)) and
\emph{relative acceleration} (TAU; expressed in \(s\)) are the
parameters of Equation~\ref{eq-velocity-time}. TAU represents the ratio
of MSS to initial acceleration (MAC; \emph{maximal acceleration},
expressed in \(ms^{-2}\)) (Equation~\ref{eq-maximal-acceleration}). Note
that TAU, given Equation~\ref{eq-velocity-time}, is the time required to
reach a velocity equal to 63.2\% of MSS.

\begin{equation}\protect\hypertarget{eq-maximal-acceleration}{}{
  MAC = \frac{MSS}{TAU}
}\label{eq-maximal-acceleration}\end{equation}

Although TAU is utilized in the equations and afterward estimated, it is
preferable to use and report MAC because it is simpler to understand,
especially for practitioners and coaches.

By deriving Equation~\ref{eq-velocity-time},
Equation~\ref{eq-acceleration-time} is obtained for horizontal
acceleration.

\begin{equation}\protect\hypertarget{eq-acceleration-time}{}{
  a(t) = \frac{MSS}{TAU} \times e^{-\frac{t}{TAU}}
}\label{eq-acceleration-time}\end{equation}

By integrating Equation~\ref{eq-velocity-time}, equation for distance
covered (Equation~\ref{eq-distance-time}) is obtained.

\begin{equation}\protect\hypertarget{eq-distance-time}{}{
  d(t) = MSS \times (t + TAU \times e^{-\frac{t}{TAU}}) - MSS \times TAU
}\label{eq-distance-time}\end{equation}

\hypertarget{model-parameters-estimation-using-timing-gates-split-times}{%
\subsection{Model parameters estimation using timing gates split
times}\label{model-parameters-estimation-using-timing-gates-split-times}}

Table~\ref{tbl-example-split-times} contains sample split times measured
during 40 \(m\) sprint performance using timing gates positioned at 5,
10, 20, 30, and 40 \(m\).

\hypertarget{tbl-example-split-times}{}
\begin{table}
\caption{\label{tbl-example-split-times}Sample split times measured during 40 m sprint performance using timing
gates positioned at 5, 10, 20, 30, and 40 m }\tabularnewline

\centering
\begin{tabular}{r|r}
\hline
Distance (m) & Split time (s)\\
\hline
5 & 1.34\\
\hline
10 & 2.06\\
\hline
20 & 3.29\\
\hline
30 & 4.44\\
\hline
40 & 5.56\\
\hline
\end{tabular}
\end{table}

To estimate model parameters using split times, distance is a
\emph{predictor} and time is the \emph{outcome} variable; hence,
Equation~\ref{eq-distance-time} takes the form of
Equation~\ref{eq-time-distance}).

\begin{equation}\protect\hypertarget{eq-time-distance}{}{
  t(d) = TAU \times W(-e^{\frac{-d}{MSS \times TAU}} - 1) + \frac{d}{MSS} + TAU
}\label{eq-time-distance}\end{equation}

\(W\) in Equation~\ref{eq-time-distance} represents \emph{Lambert's W}
function (\protect\hyperlink{ref-R-LambertW}{Goerg 2022}).
Equation~\ref{eq-distance-time}), in which time is the predictor and
distance is the outcome variable, is commonly employed in research
(\protect\hyperlink{ref-morinSpreadsheetSprintAcceleration2017}{Morin
2017};
\protect\hyperlink{ref-morinSpreadsheetSprintAcceleration2019}{Morin and
Samozino 2019};
\protect\hyperlink{ref-stenrothSpreadsheetSprintAcceleration2020}{Stenroth
and Vartiainen 2020}). This method should be avoided since reversing the
predictor, and outcome variables in a regression model may create biased
estimated parameters
(\protect\hyperlink{ref-motulskyIntuitiveBiostatisticsNonmathematical2018}{Motulsky
2018, p. 341}). This bias may not be practically significant for
profiling short sprints, but it is a statistically flawed practice and
should be avoided. It is thus preferable to utilize statistically
correct Equation~\ref{eq-time-distance}) to estimate model MSS and TAU.

Estimating MSS and TAU parameters using Equation~\ref{eq-time-distance}
as the model definition is performed using \emph{non-linear least
squares regression}. To the best of my knowledge, scientist,
researchers, and coaches have been performing short sprints modeling
using the built-in solver function of Microsoft Excel (Microsoft
Corporation, Redmond, Washington, United States)
(\protect\hyperlink{ref-samozinoSimpleMethodMeasuring2016}{Samozino et
al. 2016}; \protect\hyperlink{ref-clarkNFLCombine40Yard2017}{Clark et
al. 2017};
\protect\hyperlink{ref-morinSpreadsheetSprintAcceleration2017}{Morin
2017};
\protect\hyperlink{ref-morinSpreadsheetSprintAcceleration2019}{Morin and
Samozino 2019};
\protect\hyperlink{ref-morinSimpleMethodComputing2019}{Morin et al.
2019};
\protect\hyperlink{ref-stenrothForcevelocityProfilingIce2020}{Stenroth
et al. 2020};
\protect\hyperlink{ref-stenrothSpreadsheetSprintAcceleration2020}{Stenroth
and Vartiainen 2020}). These, and additional functionalities, have been
recently implemented in the open-source \textbf{\{shorts\}} package
(\protect\hyperlink{ref-vescoviSprintMechanicalCharacteristics2021}{Vescovi
and Jovanović 2021}; \protect\hyperlink{ref-R-shorts}{Jovanović 2022};
\protect\hyperlink{ref-jovanovic2022}{Jovanović and Vescovi 2022}) for
R-language (\protect\hyperlink{ref-R-base}{R Core Team 2022}), which
utilizes the \texttt{nlsLM()} function from the \textbf{\{minpack.lm\}}
package (\protect\hyperlink{ref-R-minpack.lm}{Elzhov et al. 2022}).
Compared to the built-in solver function of Microsoft Excel, the
\textbf{\{shorts\}} package represents a more powerful, flexible,
transparent, and reproducible environment for modeling short sprints. It
is used in this study to estimate model parameters.

Using the split times from Table~\ref{tbl-example-split-times},
estimated MSS, TAU, and MAC parameters equal to 9.02 \(ms^{-1}\), 1.14
\(s\), and 7.94 \(ms^{-2}\), respectively. \emph{Maximal relative power}
(PMAX; expressed in \(W/kg\)) is an additional parameter often estimated
and reported
(\protect\hyperlink{ref-samozinoSimpleMethodMeasuring2016}{Samozino et
al. 2016}; \protect\hyperlink{ref-morinSimpleMethodComputing2019}{Morin
et al. 2019}). PMAX is calculated using
Equation~\ref{eq-relative-power}. This method of PMAX estimation
disregards the air resistance and thus represents \emph{net} or relative
\emph{propulsive} power. Calculated PMAX using estimated MSS and MAC
parameters equal to 17.91 \(W/kg\).

\begin{equation}\protect\hypertarget{eq-relative-power}{}{
  PMAX = \frac{MSS \times MAC}{4} 
}\label{eq-relative-power}\end{equation}

\hypertarget{problems-with-parameters-estimation-using-split-times-due-to-flying-start-and-reaction-time}{%
\subsection{Problems with parameters estimation using split times due to
flying start and reaction
time}\label{problems-with-parameters-estimation-using-split-times-due-to-flying-start-and-reaction-time}}

To ensure accurate short sprint parameter estimates, the initial force
production must be synced with start time, often reffed to as ``first
movement'' triggering
(\protect\hyperlink{ref-haugenDifferenceStartImpact2012}{Haugen et al.
2012}; \protect\hyperlink{ref-haugenSprintRunningPerformance2016}{Haugen
and Buchheit 2016};
\protect\hyperlink{ref-samozinoSimpleMethodMeasuring2016}{Samozino et
al. 2016};
\protect\hyperlink{ref-haugenSprintMechanicalVariables2019}{Haugen et
al. 2019};
\protect\hyperlink{ref-haugenSprintMechanicalProperties2020}{Haugen,
Breitschädel, and Seiler 2020};
\protect\hyperlink{ref-haugenPowerForceVelocityProfilingSprinting2020}{Haugen,
Breitschädel, and Samozino 2020}). This represents a challenge when
collecting sprint data using timing gates and can substantially impact
estimated parameters.

To demonstrate this impact, imagine three hypothetical twin brothers,
Mike, Phil, and John, with the same short sprint characteristics: MSS
equal to 9 \(ms^{-1}\), TAU equal to 1.125 \(s\), MAC equal to 8
\(ms^{-2}\), and PMAX equal to 18 \(W/kg\) (these represent \emph{true}
short sprint parameters). They all perform a 40 \(m\) sprint from a
standing start using timing gates positioned at 5, 10, 20, 30, and 40
\(m\). For Mike and Phil, the timing system is activated by the initial
timing gate (i.e., when they cross the beam) at the start of the sprint
(i.e., \(d=0\) \(m\)). For John, the timing system is activated after
the gunfire.

Mike represents the \emph{theoretical model}, in which it is assumed
that the initial force production and the timing initiation are
perfectly synchronized. Mike's split have already been enlisted in
Table~\ref{tbl-example-split-times}.

On the other hand, Phil decides to move slightly behind the initial
timing gate (i.e., for 0.5 \(m\)) and use body rocking to initiate the
sprint start. In other words, Phil uses a \emph{flying start}, a common
scenario when testing field sports athletes. From a measurement
perspective, flying start distance is often recommended to avoid
premature triggering of the timing system by lifted knees or swinging
arms
(\protect\hyperlink{ref-altmannDifferentStartingDistances2015}{Altmann
et al. 2015};
\protect\hyperlink{ref-haugenSprintRunningPerformance2016}{Haugen and
Buchheit 2016};
\protect\hyperlink{ref-altmannValiditySingleBeamTiming2017}{Altmann et
al. 2017}; \protect\hyperlink{ref-altmannAccuracySingleBeam2018}{Altmann
et al. 2018};
\protect\hyperlink{ref-haugenPowerForceVelocityProfilingSprinting2020}{Haugen,
Breitschädel, and Samozino 2020}). Flying start can also result from
body rocking during the standing start. Clearly, any flying start with a
difference between the initial force production and the start time can
lead to skewed parameters and predictions. Since it is hard to get
faster at a sprint, inconsistent starts can hide the effects of the
training intervention.

Since the gunfire triggers John's start, his split times have an
additional reaction time of 0.2 \(s\). This is similar to a scenario
where the athlete prematurely triggers a timing system when standing too
close to the initial timing gate. John's data can thus be used to
demonstrate the effects of this scenario on the estimated parameters.

Timing gates utilized in this theoretical example provide precision to
two decimals (i.e., closest 10 \(ms)\), representing a measurement error
source. A graphical representation of the sprint splits can be found in
Figure~\ref{fig-mike-phil-john-split-times}.

\begin{figure}

{\centering \includegraphics[width=0.9\textwidth,height=\textheight]{shorts-simulation-paper_files/figure-pdf/fig-mike-phil-john-split-times-1.pdf}

}

\caption{\label{fig-mike-phil-john-split-times}Phil, Mike, and John
split times over a 40 \(m\) distance. All three brothers have identical
sprint performances but utilize different sprint starts, resulting in
different split times.}

\end{figure}

Estimated sprint parameters can be found in
Table~\ref{tbl-mike-phil-john-est-params}). As seen from the results
(Table~\ref{tbl-mike-phil-john-est-params})), estimated short sprint
parameters for all three brothers differ from the \emph{true} parameters
used to generate the data (i.e., their \emph{true} short sprint
characteristics). All three brothers have a bias in estimated parameters
due to timing gates' precision to 2 decimals (i.e., 10 \(ms\)). Bias in
estimated parameters in Phil's case is due to the flying start involved,
while in John's case, it is due to the reaction time involved in the
split times.

\hypertarget{tbl-mike-phil-john-est-params}{}
\begin{table}

\caption{\label{tbl-mike-phil-john-est-params}Estimated sprint parameters for Mike, Phil, and John. All three brothers
have identical sprint performance but utilize different sprint starts,
which results in different split times, and thus different sprint
parameter estimates. Due to the timing gates' precision to 2 decimals
(i.e., 10 ms), estimated Mike's parameters also differ from the true
values }(ref:mike-phil-john-est-params)}
\centering
\begin{tabular}[t]{lrrrr}
\toprule
Athlete & MSS & TAU & MAC & PMAX\\
\midrule
True & 9.00 & 1.12 & 8.00 & 18.0\\
Mike (theoretical) & 9.02 & 1.14 & 7.94 & 17.9\\
Phil (flying start) & 8.60 & 0.61 & 14.00 & 30.1\\
John (gunfire) & 9.59 & 1.62 & 5.93 & 14.2\\
\bottomrule
\end{tabular}
\end{table}

\hypertarget{how-to-overcome-missing-the-initial-force-production-when-using-timing-gates}{%
\subsection{How to overcome missing the initial force production when
using timing
gates?}\label{how-to-overcome-missing-the-initial-force-production-when-using-timing-gates}}

The literature suggests using a correction factor of +0.5 \(s\) as a
viable solution (i.e., simply adding +0.5 \(s\) to split times) to
convert to ``first movement'' triggering when utilizing recommended 0.5
\(m\) flying distance behind the initial timing gate
(\protect\hyperlink{ref-haugenDifferenceStartImpact2012}{Haugen et al.
2012}; \protect\hyperlink{ref-haugenSprintRunningPerformance2016}{Haugen
and Buchheit 2016};
\protect\hyperlink{ref-haugenSprintMechanicalVariables2019}{Haugen et
al. 2019};
\protect\hyperlink{ref-haugenSprintMechanicalProperties2020}{Haugen,
Breitschädel, and Seiler 2020}). Intriguingly, the average difference
between the standing start with a photocell trigger and a block start to
gunfire for a 40-meter sprint was 0.27 \(s\)
(\protect\hyperlink{ref-haugenDifferenceStartImpact2012}{Haugen et al.
2012}). Consequently, although a timing correction factor is required to
prevent further inaccuracies in estimates of kinetic variables (e.g.,
overestimate power), a correction factor that is too big would have the
opposite effect (e.g., underestimate power).

\hypertarget{the-estimated-time-correction-model}{%
\subsubsection{The Estimated time correction
model}\label{the-estimated-time-correction-model}}

Instead of using \emph{apriori} time correction from the literature,
this parameter may be estimated using the supplied data, together with
MSS and TAU. Stenroth et al.
(\protect\hyperlink{ref-stenrothForcevelocityProfilingIce2020}{2020})
propose the same approach, titled the \emph{time shift method}, and the
estimated parameter, named the \emph{time shift parameter}. In
accordance with the current literature, this parameter is termed
\emph{time correction} (TC)
(\protect\hyperlink{ref-vescoviSprintMechanicalCharacteristics2021}{Vescovi
and Jovanović 2021}).

Using the original Equation~\ref{eq-time-distance} to implement the TC
parameter now yields the new Equation~\ref{eq-time-correction}.
Equation~\ref{eq-time-correction} is utilized as the model definition in
the \emph{Estimated TC} model, as opposed to the model using
Equation~\ref{eq-time-distance}, which is termed the \emph{No
correction} model in this study. The model in which TC is fixed (i.e.,
by simply adding TC to split times) is termed the \emph{Fixed TC} model.

\begin{equation}\protect\hypertarget{eq-time-correction}{}{
  t(d) = TAU \times W(-e^{\frac{-d}{MSS \times TAU}} - 1) + \frac{d}{MSS} + TAU - TC 
}\label{eq-time-correction}\end{equation}

From a regression perspective, the TC parameter can be viewed as an
\emph{intercept}. It can be beneficial when assuming a fixed time shift
is involved (i.e., reaction time or premature triggering of the timing
equipment). Comparing the split times of Mike and John in
Figure~\ref{fig-mike-phil-john-split-times}, it can be noticed that the
lines are parallel. In this scenario, the \emph{Estimated TC} model can
remove bias between Mike and John. The \emph{Estimated TC} model can
also help remove bias in estimated parameters in Phil's case. However,
when looking closely at Figure~\ref{fig-mike-phil-john-split-times}, it
can be noticed that Phil's and Mike's lines are not parallel. This is
because there is already some velocity when the initial timing gate is
triggered; thus, the time shift is not constant.

These models (i.e., \emph{Fixed TC} of +0.3, +0.5 \(s\), and
\emph{Estimate TC} model) are applied to Mike, Phil, and John's split
times. The estimated model parameters can be found in
Table~\ref{tbl-all-estimates}) and previously estimated parameter values
using the \emph{No correction} model. As can be noted from
Table~\ref{tbl-all-estimates}, adding +0.3 \(s\) worked well for Phil in
terms of approaching \emph{true} parameter values, while adding +0.5
\(s\) was detrimental in un-biasing estimated parameters.

The \emph{Estimated TC} model worked well for all three athletes in
terms of un-biasing the parameter estimates. The estimated TC parameter
for John was also very close to the \emph{true} reaction time of 0.2
\(s\).

\hypertarget{estimated-flying-distance-model}{%
\subsubsection{Estimated flying distance
model}\label{estimated-flying-distance-model}}

Although the \emph{Estimated TC} model performed well in Phil's case
(twin brother doing flying start), instead of assuming time shift (which
helps in un-biasing the estimates compared to the \emph{No correction}
model), the model definition that assumes \emph{flying start distance}
(FD) involved in the \emph{data-generating-process} (DGP) can be
utilized. This \emph{Estimated FD} model utilizes
Equation~\ref{eq-distance-correction} as the model definition.

\begin{equation}\protect\hypertarget{eq-distance-correction}{}{
\begin{split}
   t(d) &= (TAU \times W(-e^{\frac{-d + FD}{MSS \times TAU}} - 1) + \frac{d + FD}{MSS} + TAU) \\ 
   &\quad-(TAU \times W(-e^{\frac{FD}{MSS \times TAU}} - 1) + \frac{FD}{MSS} + TAU) 
\end{split}
}\label{eq-distance-correction}\end{equation}

Table~\ref{tbl-all-estimates} contains all model estimates for three
brothers, including the \emph{Estimated FD} model. It can be noticed
that the \emph{Estimated FD} model unbiased estimates for Phil but
failed to be estimated for John (brother that starts at gunfire and has
reaction time involved in his split times). This is because the
\emph{Estimated FD} model is \emph{ill-defined} under that scenario and
cannot have a \emph{negative} flying distance.

Overall, each model definition has assumed the mechanism of the data
generation. \emph{No correction} model assumes perfect synchronization
of the sprint initiation with the start of the timing. The
\emph{Estimated TC} model introduces a simple intercept that can help
estimate parameters when an assumed time shift is involved (e.g., when
reaction time is involved or premature triggering of the initial timing
gate). \emph{Estimated TC} can also be used when flying start is
utilized, but it assumes the constant time shift, which is not the case
in that scenario due to already gained velocity at the start. The
\emph{Estimated FD} model assumes there is a flying sprint involved in
the DGP and, as shown in Table~\ref{tbl-all-estimates}, can be
ill-defined when there is no flying distance involved, but there is a
time shift. All three models assume athlete accelerates according to the
mono-exponential Equation~\ref{eq-velocity-time}.

This work aims to explore the behavior of these three models under
simulated and known conditions. This is needed to provide a theoretical
understanding of the limits and expected errors of the short sprints
modeling, which can later inform more practical studies involving
athletes.

\hypertarget{tbl-all-estimates}{}
\begin{table}

\caption{\label{tbl-all-estimates}Estimated sprint parameters for Mike, Phil, and John using No
correction, Fixed time corrections (TC), Estimated TC, and Estimated FD
models. }(ref:all-estimates)}
\centering
\begin{tabular}[t]{llrrrrrr}
\toprule
Model & Athlete & MSS & TAU & MAC & PMAX & TC & FD\\
\midrule
True & True & 9.00 & 1.12 & 8.00 & 18.0 &  & \\
No correction & Mike (theoretical) & 9.02 & 1.14 & 7.94 & 17.9 &  & \\
No correction & Phil (flying start) & 8.60 & 0.61 & 14.00 & 30.1 &  & \\
No correction & John (gunfire) & 9.59 & 1.62 & 5.93 & 14.2 &  & \\
Fixed +0.3s TC & Mike (theoretical) & 10.01 & 1.93 & 5.19 & 13.0 &  & \\
\addlinespace
Fixed +0.3s TC & Phil (flying start) & 9.05 & 1.13 & 8.02 & 18.2 &  & \\
Fixed +0.3s TC & John (gunfire) & 11.29 & 2.79 & 4.05 & 11.4 &  & \\
Fixed +0.5s TC & Mike (theoretical) & 11.29 & 2.79 & 4.05 & 11.4 &  & \\
Fixed +0.5s TC & Phil (flying start) & 9.62 & 1.61 & 5.98 & 14.4 &  & \\
Fixed +0.5s TC & John (gunfire) & 13.67 & 4.26 & 3.21 & 11.0 &  & \\
\addlinespace
Estimated TC & Mike (theoretical) & 9.04 & 1.15 & 7.86 & 17.8 & 0.01 & \\
Estimated TC & Phil (flying start) & 9.00 & 1.08 & 8.35 & 18.8 & 0.28 & \\
Estimated TC & John (gunfire) & 9.04 & 1.15 & 7.86 & 17.8 & -0.19 & \\
Estimated FD & Mike (theoretical) & 9.04 & 1.15 & 7.86 & 17.8 &  & 0.00\\
Estimated FD & Phil (flying start) & 9.03 & 1.16 & 7.82 & 17.7 &  & 0.54\\
\addlinespace
Estimated FD & John (gunfire) &  &  &  &  &  & \\
\bottomrule
\end{tabular}
\end{table}

\hypertarget{methods}{%
\section{Methods}\label{methods}}

\hypertarget{simulation-design}{%
\subsection{Simulation design}\label{simulation-design}}

In this simulation, data is generated using \emph{true} MSS (ranging
from 7 to 11 \(ms^-1\), in increments of 0.05 \(ms^-1\), resulting in a
total of 81 unique values), MAC (ranging from 7 to 11 \(ms^-2\), in
increments of 0.05 \(ms^-2\), resulting in a total of 81 unique values),
and flying distance (ranging from 0 to 0.5 \(m\), in increments of 0.01
\(m\), resulting in a total of 51 unique values). Each flying sprint
distance consists of 6,561 MSS and MAC combinations.

Split times are estimated using timing gates positioned at 5, 10, 20,
30, and 40 \(m\), with the rounding to the closest 10 \(ms\). In total,
there are 334,611 sprints simulated.

\hypertarget{statistical-analysis}{%
\subsection{Statistical analysis}\label{statistical-analysis}}

MSS, MAC, TAU, and PMAX are estimated for each simulated sprint using
the \emph{No correction}, \emph{Estimated TC}, and \emph{Estimated FD
models}. The agreement between \emph{true} and estimated parameter
values is evaluated using the \emph{percent difference} (\(\%Diff\))
estimator (Equation~\ref{eq-percent-difference}).

\begin{equation}\protect\hypertarget{eq-percent-difference}{}{
  \%Diff = 100 \times \frac{estimated - true}{true}
}\label{eq-percent-difference}\end{equation}

The distribution of the simulated \(\%Diff\) is summarized using
\(median\) and 95\% \emph{highest-density continuous interval}
(\(HDCI\))
(\protect\hyperlink{ref-kruschkeDoingBayesianData2015}{Kruschke 2015};
\protect\hyperlink{ref-kruschkeBayesianDataAnalysis2018}{Kruschke and
Liddell 2018a};
\protect\hyperlink{ref-kruschkeBayesianNewStatistics2018}{Kruschke and
Liddell 2018b};
\protect\hyperlink{ref-kruschkeRejectingAcceptingParameter2018}{Kruschke
2018};
\protect\hyperlink{ref-makowskiBayestestRDescribingEffects2019}{Makowski
et al. 2019}).

To provide magnitude interpretation of the \(\%Diff\), \emph{region of
practical equivalence} (\(ROPE\)), as well as the proportion of the
simulations that lie within \(ROPE\) (\(inside \; ROPE\); expressed as a
percentage)
(\protect\hyperlink{ref-kruschkeDoingBayesianData2015}{Kruschke 2015};
\protect\hyperlink{ref-kruschkeBayesianDataAnalysis2018}{Kruschke and
Liddell 2018a};
\protect\hyperlink{ref-kruschkeBayesianNewStatistics2018}{Kruschke and
Liddell 2018b};
\protect\hyperlink{ref-kruschkeRejectingAcceptingParameter2018}{Kruschke
2018};
\protect\hyperlink{ref-makowskiBayestestRDescribingEffects2019}{Makowski
et al. 2019};
\protect\hyperlink{ref-jovanovicBmbstatsBootstrapMagnitudebased2020}{Jovanović
2020}), are calculated. For the purpose of this paper, \(ROPE\) is
assumed to be equal to 95\% \(HDCI\) of the \(\%Diff\) using the
\emph{No correction} model and no flying distance. Theoretically,
\(ROPE\) represents the lowest error (i.e., the best agreement) that can
be achieved. It is limited purely by the timing gates measurement
precision (i.e., rounding to the closest 10 \(ms\)) and simulated
parameters.

In addition to estimating agreement between \emph{true} and estimated
parameter values, practitioners are often interested in whether they can
use estimated measures to track changes in the \emph{true} measures. A
\emph{minimal detectable change} estimator with 95\% confidence
(\(\%MDC_{95}\))
(\protect\hyperlink{ref-furlanApplicabilityStandardError2018}{Furlan and
Sterr 2018};
\protect\hyperlink{ref-jovanovicBmbstatsBootstrapMagnitudebased2020}{Jovanović
2020}) is utilized to estimate this precision. The \(\%MDC_{95}\) value
might be regarded as the minimum amount of change that needs to be
observed in the estimated parameter for it to be considered a
\emph{true} change.

In this study, \(\%MDC_{95}\) is calculated using \emph{percent residual
standard error} (\(\%RSE\); Equation~\ref{eq-percent-rse}) of the linear
regression between \emph{true} (predictor) and estimated parameter
values (outcome) (Equation~\ref{eq-smallest-detectable-change}). Since
simulated data with the known \emph{true} values are utilized, \(\%RSE\)
represents the \emph{percent standard error of the measurement}
(\(\%SEM\)) in the estimated parameters.

\begin{equation}\protect\hypertarget{eq-percent-rse}{}{
  \%RSE = \sqrt{\frac{\sum_{i=1}^N{(100 \times \frac{y_i - \hat{y_i}}{\hat{y_i}})^2}}{N-2}}
}\label{eq-percent-rse}\end{equation}

\begin{equation}\protect\hypertarget{eq-smallest-detectable-change}{}{
  \%MDC_{95} = \%RSE \times \sqrt{2} \times 1.96
}\label{eq-smallest-detectable-change}\end{equation}

In addition to providing \(\%MDC_{95}\) for the estimated parameters,
the lowest \(\%MDC_{95}\) is estimated using the \emph{No correction}
model and no flying distance (\(\%MDC_{95}^{lowest}\)). Theoretically,
\(\%MDC_{95}^{lowest}\) represents the lowest \(\%MDC_{95}\) that can be
achieved, and it is limited purely by the timing gates measurement
precision (i.e., rounding to the closest 10 \(ms\)) and simulated
parameters. \(\%MDC_{95}^{lowest}\) is used only as a reference to
evaluate estimated parameters' \(\%MDC_{95}\).

The analyses, as mentioned earlier, are performed on both \emph{pooled}
dataset (i.e., using all flying distance) and across every flying
distance. It is hypothesized that the \emph{Estimated FD} model will
have the highest \(inside \; ROPE\) estimates and the lowest
\(\%MDC_{95}\) estimates.

Statistical analyses and graph construction were performed using the
software R 4.2.1 (\protect\hyperlink{ref-R-base}{R Core Team 2022}) in
RStudio (version 2022.07.1+554).

\hypertarget{results}{%
\section{Results}\label{results}}

\hypertarget{model-fitting}{%
\subsection{Model fitting}\label{model-fitting}}

Table~\ref{tbl-not-fitted} contains failed model fitting for the
\emph{Estimated FD} model. These were disregarded from further analysis.

The reason for these failed model fittings is probably the combination
of the very small flying distance and the measurement precision of the
timing gates, resulting in an ill-defined model that cannot be fitted.

\hypertarget{tbl-not-fitted}{}
\begin{table}
\caption{\label{tbl-not-fitted}Failed model fittings for the Estimated FD model }\tabularnewline

\centering
\begin{tabular}{rrrr}
\toprule
Flying distance (m) & Not fitted & Total & Not fitted (\%)\\
\midrule
0.00 & 1765 & 6561 & 26.90\\
0.01 & 12 & 6561 & 0.18\\
0.02 & 16 & 6561 & 0.24\\
0.03 & 10 & 6561 & 0.15\\
0.04 & 4 & 6561 & 0.06\\
\addlinespace
0.05 & 1 & 6561 & 0.02\\
\bottomrule
\end{tabular}
\end{table}

\hypertarget{percent-difference}{%
\subsection{Percent difference}\label{percent-difference}}

\hypertarget{region-of-practical-equivalence}{%
\subsubsection{Region of practical
equivalence}\label{region-of-practical-equivalence}}

Estimated ROPEs are equal to -0.3 to 0.33\% for MSS, -0.73 to 0.74\% for
MAC, -1.03 to 1\% for TAU, and -0.5 to 0.5\% for PMAX
(Table~\ref{tbl-ROPE-pooled}) and grey horizontal bars in
Figure~\ref{fig-graph-ROPE-pooled} and Figure~\ref{fig-graph-per-FD}. An
interesting finding is that, given simulation parameters (particularly
the precision of the timing gates to the closest 10 \(ms\)), MSS has the
lowest \(ROPE\) compared to other short sprint parameters. Since
\(ROPE\) represents the lowest estimation error, MSS is the parameter
that could be, given this theoretical simulation, estimated with the
most precision. In contrast, TAU and MAC can be estimated with the least
precision.

\hypertarget{pooled-analysis}{%
\subsubsection{Pooled analysis}\label{pooled-analysis}}

The pooled analysis is performed using all flying distances pooled
together. As such, the pooled analysis represents the overall estimate
of the agreement between \emph{true} and estimated parameter values
across simulated conditions.

Figure~\ref{fig-graph-ROPE-pooled} depicts the distribution of the
pooled \(\%Diff\). As expected, the \emph{Estimated FD} model performed
with the highest \(inside \; ROPE\) parameter values (from 20 to 72\%),
with the most narrow 95\% \(HDCI\)s (from -5 to 5\%), and no bias
involved.

On the other hand, the \emph{No correction} model performed poorly, with
the lowest inside \(ROPE\) parameter values (from 2 to 2\%), with the
widest 95\% \(HDCI\)s (from -46 to 80\%), and with the apparent bias
indicated with the \(median\) parameter values being outside of \(ROPE\)
(from -35 to 49\%). In addition, visual inspection of
Figure~\ref{fig-graph-ROPE-pooled} indicates a \emph{non-normal}
distribution of estimated \(\%Diff\) parameter values, demanding further
analysis across flying distance values.

The \emph{Estimated TC} model performed similarly to the \emph{Estimated
FD} model with a slightly lower \(inside \; ROPE\) parameter values
(from 9 to 67\%), wider 95\% \(HDCI\)s (from -9 to 8\%), and with
obvious bias, although much smaller than the \emph{No correction} model
bias (from -3 to 3\%).

Table~\ref{tbl-ROPE-pooled} contains the pooled analysis results summary
for every model and short sprint parameter.

\begin{figure}

{\centering \includegraphics[width=0.9\textwidth,height=\textheight]{shorts-simulation-paper_files/figure-pdf/fig-graph-ROPE-pooled-1.pdf}

}

\caption{\label{fig-graph-ROPE-pooled}Pooled distribution of the
\(\%Diff\). Error bars represent the distribution \(median\) and 95\%
\(HDCI\). A grey area represents parameter \(ROPE\) (assumed to be equal
to 95\% \(HDCI\) of the \(\%Diff\) using \emph{No correction} model and
no flying distance).}

\end{figure}

\hypertarget{tbl-ROPE-pooled}{}
\begin{table}

\caption{\label{tbl-ROPE-pooled}ROPEs, a summary of \%Diff distribution, and inside ; ROPE for pooled
analysis. }(ref:tbl-ROPE-pooled)}
\centering
\begin{tabular}[t]{lrlrr}
\toprule
Parameter & ROPE (\%) & Model & \% Diff & Inside ROPE (\%)\\
\midrule
MSS & -0.3 to 0.33\% & No correction & median -3\%, 95\% HDCI [-7 to 0\%] & 2\%\\
MSS & -0.3 to 0.33\% & Estimated TC & median 0\%, 95\% HDCI [-1 to 0\%] & 67\%\\
MSS & -0.3 to 0.33\% & Estimated FD & median 0\%, 95\% HDCI [-1 to 1\%] & 72\%\\
MAC & -0.73 to 0.74\% & No correction & median 49\%, 95\% HDCI [11 to 80\%] & 2\%\\
MAC & -0.73 to 0.74\% & Estimated TC & median 3\%, 95\% HDCI [-2 to 8\%] & 12\%\\
\addlinespace
MAC & -0.73 to 0.74\% & Estimated FD & median 0\%, 95\% HDCI [-4 to 4\%] & 25\%\\
TAU & -1.03 to 1\% & No correction & median -35\%, 95\% HDCI [-46 to -11\%] & 2\%\\
TAU & -1.03 to 1\% & Estimated TC & median -3\%, 95\% HDCI [-9 to 2\%] & 16\%\\
TAU & -1.03 to 1\% & Estimated FD & median 0\%, 95\% HDCI [-5 to 5\%] & 31\%\\
PMAX & -0.5 to 0.5\% & No correction & median 44\%, 95\% HDCI [6 to 73\%] & 2\%\\
\addlinespace
PMAX & -0.5 to 0.5\% & Estimated TC & median 3\%, 95\% HDCI [-2 to 8\%] & 9\%\\
PMAX & -0.5 to 0.5\% & Estimated FD & median 0\%, 95\% HDCI [-4 to 4\%] & 20\%\\
\bottomrule
\end{tabular}
\end{table}

\hypertarget{analysis-across-flying-distances}{%
\subsubsection{Analysis across flying
distances}\label{analysis-across-flying-distances}}

Figure~\ref{fig-graph-per-FD} depicts the analysis results for every
flying distance in the simulation. \(inside \; ROPE\) parameter
estimates are calculated and depicted in
Figure~\ref{fig-graph-inside-ROPE} for easier comprehension.

As expected, the \emph{No correction} model demonstrated increasing bias
as the flying distance increases (from -46 to 76\%), the widest 95\%
\(HDCI\)s (from -47 to 84\%), and the lowest \(inside \; ROPE\)
estimated parameter values.

\emph{Estimated TC} showed a small bias trend across flying distances
(from -6 to 6\%), resulting in decreasing \(inside \; ROPE\) performance
(from 0 to 75\%; see Figure~\ref{fig-graph-inside-ROPE}), although with
much smaller 95\% \(HDCI\)s (from -10 to 11\%) compared to \emph{No
correction} model.

\emph{Estimated FD}, as hypothesized, showed no bias and thus a stable
\(inside \; ROPE\) performance across flying distances (see
Figure~\ref{fig-graph-inside-ROPE}), with minimal 95\% \(HDCI\)s (from
-5 to 6\%).

\begin{figure}

{\centering \includegraphics[width=0.9\textwidth,height=\textheight]{shorts-simulation-paper_files/figure-pdf/fig-graph-per-FD-1.pdf}

}

\caption{\label{fig-graph-per-FD}Distribution of the \(\%Diff\) across
every flying distance in the simulation. Error bars represent the
distribution \(median\) and 95\% \(HDCI\). A grey area represents
parameter \(ROPE\) (assumed to be equal to 95\% \(HDCI\) of the
\(\%Diff\) using \emph{No correction} model and no flying distance). For
the less crowded visualization, flying distance in increments of 0.05
\(m\) is plotted.}

\end{figure}

\begin{figure}

{\centering \includegraphics[width=0.9\textwidth,height=\textheight]{shorts-simulation-paper_files/figure-pdf/fig-graph-inside-ROPE-1.pdf}

}

\caption{\label{fig-graph-inside-ROPE}\(inside \; ROPE\) estimated
across every flying distance in the simulation.}

\end{figure}

\hypertarget{minimal-detectable-change}{%
\subsection{Minimal detectable change}\label{minimal-detectable-change}}

\hypertarget{lowest-minimum-detectable-change}{%
\subsubsection{Lowest Minimum Detectable
Change}\label{lowest-minimum-detectable-change}}

Estimated \(\%MDCs_{95}^{lowest}\) is equal to 0.45\% for MSS, 1.06\%
for MAC, 1.47\% for TAU, and 0.7\% for PMAX (column \emph{lowest} in
Table~\ref{tbl-pooled-MDC}) and dashed grey horizontal lines in
Figure~\ref{fig-graph-MDC}). An interesting finding is that, given
simulation parameters (particularly the precision of the timing gates to
the closest 10 \(ms\)), MSS has the lowest \(\%MDCs_{95}^{lowest}\)
compared to other short sprint parameters. Since
\(\%MDCs_{95}^{lowest}\) represents the lowest minimal detectable
change, MSS is the parameter whose change could be, given this
theoretical simulation, estimated with the most precision. In contrast,
TAU and MAC changes can be estimated with the least precision.

\hypertarget{pooled-analysis-1}{%
\subsubsection{Pooled analysis}\label{pooled-analysis-1}}

Pooled \(\%MDCs_{95}\) represents an estimate of the \emph{sensitivity}
to detect \emph{true} change with 95\% confidence when the flying start
distance is not standardized (but within simulation parameter limits
(ranging from 0 to 0.5 \(m\)). As expected, the \emph{No correction}
model demonstrates the highest \(\%MDCs_{95}\) (from 3 to 44\%), while
\emph{Estimated TC} and \emph{Estimated FD} demonstrated much smaller
\(\%MDCs_{95}\) (from 1 to 8\% and from 1 to 7\%, respectively)
(Table~\ref{tbl-pooled-MDC}).

An interesting finding is that the MSS parameter showed very low
\(\%MDCs_{95}\) across models (from 1 to 3\%), even for the \emph{No
correction} model. This indicates that even the non-standardized short
sprint monitoring (i.e., without standardized flying distance) using the
\emph{No correction} model, given simulation parameters, can be used to
track changes in MSS. TAU, MAC, and PMAX parameters, on the other hand,
demand a much larger \(\%MDCs_{95}\) (from 7 to 44\%, from 6 to 37\%,
and from 6 to 36\%, respectively).

\hypertarget{tbl-pooled-MDC}{}
\begin{table}
\caption{\label{tbl-pooled-MDC}Pooled \%MDCs\_\{95\} estimated using pooled simulation dataset. }\tabularnewline

\centering
\begin{tabular}{lrrrl}
\toprule
Parameter & lowest & No correction & Estimated TC & Estimated FD\\
\midrule
MSS & 0.45 \% & 3 \% & 1 \% & 1 \%\\
MAC & 1.06 \% & 37 \% & 7 \% & 6 \%\\
TAU & 1.47 \% & 44 \% & 8 \% & 7 \%\\
PMAX & 0.7 \% & 36 \% & 7 \% & 6 \%\\
\bottomrule
\end{tabular}
\end{table}

\hypertarget{analysis-across-flying-distances-1}{%
\subsubsection{Analysis across flying
distances}\label{analysis-across-flying-distances-1}}

When estimated across flying distances, \(\%MDCs_{95}\) shows
interesting and surprising patterns (Figure~\ref{fig-graph-MDC})). For
every short sprint parameter, \emph{Estimated TC} showed stable and
lower \(\%MDCs_{95}\) compared to \emph{Estimated FD} (from 1 to 6\% and
from 1 to 8\%, respectively). This is surprising because even if it has
biased estimates of short sprint parameters (see
\protect\hyperlink{percent-difference}{Percent difference} results
section, mainly Figure~\ref{fig-graph-inside-ROPE})) compared to the
\emph{Estimated FD}, \emph{Estimated TC} might be more sensitive to
detect \emph{changes}, given simulation parameters.

Another surprising finding is that the \emph{No correction} model, even
if shown to be highly biased in estimating short sprint parameter values
(see \protect\hyperlink{percent-difference}{Percent difference} results
section, mainly Figure~\ref{fig-graph-per-FD}), showed the lowest
\(\%MDCs_{95}\) for the MAC and TAU parameters (from 1 to 5\% and from 1
to 3\% respectively). This indicates that when short sprint measurement
is standardized (i.e., athletes perform with the same flying distance),
given the simulation parameters, the \emph{No correction} model can be
the most sensitive model to detect \emph{changes} in MAC and TAU
parameters. This is the case for the MSS and PMAX parameters (from 0 to
3\% and from 1 to 9\%, respectively).

When it comes to estimating \emph{changes} in short sprint parameters,
\emph{change} in MSS is the most sensitive to be detected (from 0 to
3\%) compared to MAC (from 1 to 7\%), TAU (from 1 to 8\%), and PMAX
(from 1 to 9\%).

\begin{figure}

{\centering \includegraphics[width=0.9\textwidth,height=\textheight]{shorts-simulation-paper_files/figure-pdf/fig-graph-MDC-1.pdf}

}

\caption{\label{fig-graph-MDC}Estimated \(\%MDCs_{95}\) across every
flying distance in the simulation. The dashed line represents
\(\%MDCs_{95}^{lowest}\)}

\end{figure}

\hypertarget{conclusion}{%
\section{Conclusion}\label{conclusion}}

The simulation study employed in this paper demonstrated some expected
and unexpected theoretical findings. Among the expected findings are (1)
the bias and low \(inside \; ROPE\) performance in estimating short
sprint parameters using the \emph{No correction} model, (2) more
negligible bias and higher \(inside \; ROPE\) for the \emph{Estimated
TC} model, and (3) no bias and highest \(inside \; ROPE\) for the
\emph{Estimated FD} model.

The unexpected finding of this study is the performance of the \emph{No
correction} model in sensitivity of estimating the \emph{change} of the
MAC and TAU parameters, which outperformed the other two models.

When estimating short sprint parameters across models, given simulation
parameters, MSS and \emph{change} in MSS can be estimated more precisely
compared to TAU, MAC, and PMAX parameters and their \emph{changes}.

In addition to model performances, this simulation study provided the
\(ROPE\)s and \(\%MDCs_{95}^{lowest}\). These could be useful for
further validity and reliability studies evaluating short sprint model
performance involving \emph{real} athletes and timing gates positioned
at the exact distances with the exact rounding.

The takeaway message for the practitioners is that besides standardizing
the sprint starting technique for the short sprint performance
monitoring, it would be wise to utilize and track the results of all
three models. The \emph{Estimated FD} model will provide unbiased
estimates of the current performance, but the \emph{No correction} model
might be more sensitive in detecting changes in TAU and MAC parameters.

This practical conclusion should be taken with caution since it is based
on the results of this theoretical simulation. Additional studies
involving \emph{real} athletes in evaluating the performance of these
three models are needed. These should involve estimating the short
sprint parameters agreement between gold-standard (i.e.,
\emph{criterion}) measure (e.g., radar gun, laser gun, or video
analysis) and \emph{practical} timing gates measure. One such study is
currently in preparation. In addition to theoretical findings, such a
study will provide model performance estimates when biological
variability is involved in short sprints, which is not considered in the
current study.

\hypertarget{supplemental-material}{%
\section{Supplemental material}\label{supplemental-material}}

The \emph{Quarto} (\protect\hyperlink{ref-Allaire_Quarto_2022}{Allaire
et al. 2022}) source code for this paper and analysis can be found on
the GitHub repository:
\url{https://github.com/mladenjovanovic/shorts-simulation-paper}.

\hypertarget{references}{%
\section*{References}\label{references}}
\addcontentsline{toc}{section}{References}

\hypertarget{refs}{}
\begin{CSLReferences}{0}{0}
\leavevmode\vadjust pre{\hypertarget{ref-Allaire_Quarto_2022}{}}%
Allaire JJ, Teague C, Scheidegger C, Xie Y, Dervieux C. 2022. {Quarto}
{[}Internet{]}. {[}place unknown{]}.
\url{https://doi.org/10.5281/zenodo.5960048}

\leavevmode\vadjust pre{\hypertarget{ref-altmannDifferentStartingDistances2015}{}}%
Altmann S, Hoffmann M, Kurz G, Neumann R, Woll A, Haertel S. 2015.
Different {Starting Distances Affect} 5-m {Sprint Times}. Journal of
Strength and Conditioning Research {[}Internet{]}. {[}accessed 2021 Jun
21{]} 29(8):2361--2366.
\url{https://doi.org/10.1519/JSC.0000000000000865}

\leavevmode\vadjust pre{\hypertarget{ref-altmannValiditySingleBeamTiming2017}{}}%
Altmann S, Spielmann M, Engel FA, Neumann R, Ringhof S, Oriwol D,
Haertel S. 2017. Validity of {Single}-{Beam Timing Lights} at {Different
Heights}. Journal of Strength and Conditioning Research {[}Internet{]}.
{[}accessed 2021 Jun 21{]} 31(7):1994--1999.
\url{https://doi.org/10.1519/JSC.0000000000001889}

\leavevmode\vadjust pre{\hypertarget{ref-altmannAccuracySingleBeam2018}{}}%
Altmann S, Spielmann M, Engel FA, Ringhof S, Oriwol D, Härtel S, Neumann
R. 2018. Accuracy of single beam timing lights for determining
velocities in a flying 20-m sprint: {Does} timing light height matter?
jhse {[}Internet{]}. {[}accessed 2021 Jun 21{]} 13(3).
\url{https://doi.org/10.14198/jhse.2018.133.10}

\leavevmode\vadjust pre{\hypertarget{ref-brownAssessmentLinearSprinting2004}{}}%
Brown TD, Vescovi JD, Vanheest JL. 2004.
\href{https://www.ncbi.nlm.nih.gov/pmc/articles/PMC3938058}{Assessment
of linear sprinting performance: A theoretical paradigm}. Journal of
Sports Science \& Medicine. 3(4):203--210.

\leavevmode\vadjust pre{\hypertarget{ref-buchheitMechanicalDeterminantsAcceleration2014}{}}%
Buchheit M, Samozino P, Glynn JA, Michael BS, Al Haddad H,
Mendez-Villanueva A, Morin JB. 2014. Mechanical determinants of
acceleration and maximal sprinting speed in highly trained young soccer
players. Journal of Sports Sciences. 32(20):1906--1913.
\url{https://doi.org/10.1080/02640414.2014.965191}

\leavevmode\vadjust pre{\hypertarget{ref-clarkNFLCombine40Yard2017}{}}%
Clark KP, Rieger RH, Bruno RF, Stearne DJ. 2017. The {NFL Combine}
40-{Yard Dash}: {How Important} is {Maximum Velocity}? Journal of
Strength and Conditioning Research.:1.
\url{https://doi.org/10.1519/JSC.0000000000002081}

\leavevmode\vadjust pre{\hypertarget{ref-edwardsSprintAccelerationCharacteristics2020}{}}%
Edwards T, Piggott B, Banyard HG, Haff GG, Joyce C. 2020. Sprint
acceleration characteristics across the {Australian} football
participation pathway. Sports Biomechanics.:1--13.
\url{https://doi.org/10.1080/14763141.2020.1790641}

\leavevmode\vadjust pre{\hypertarget{ref-R-minpack.lm}{}}%
Elzhov TV, Mullen KM, Spiess A-N, Bolker B. 2022. Minpack.lm: R
interface to the levenberg-marquardt nonlinear least-squares algorithm
found in MINPACK, plus support for bounds {[}Internet{]}. {[}place
unknown{]}. \url{https://CRAN.R-project.org/package=minpack.lm}

\leavevmode\vadjust pre{\hypertarget{ref-furlanApplicabilityStandardError2018}{}}%
Furlan L, Sterr A. 2018. The {Applicability} of {Standard Error} of
{Measurement} and {Minimal Detectable Change} to {Motor Learning
Research}---{A Behavioral Study}. Front Hum Neurosci {[}Internet{]}.
{[}accessed 2022 Jul 15{]} 12:95.
\url{https://doi.org/10.3389/fnhum.2018.00095}

\leavevmode\vadjust pre{\hypertarget{ref-doi:10.1098ux2frspb.1927.0035}{}}%
Furusawa K, Hill AV, Parkinson JL. 1927. The dynamics of "sprint"
running. Proceedings of the Royal Society of London Series B, Containing
Papers of a Biological Character. 102(713):29--42.
\url{https://doi.org/10.1098/rspb.1927.0035}

\leavevmode\vadjust pre{\hypertarget{ref-R-LambertW}{}}%
Goerg GM. 2022. LambertW: Probabilistic models to analyze and
gaussianize heavy-tailed, skewed data {[}Internet{]}. {[}place
unknown{]}. \url{https://CRAN.R-project.org/package=LambertW}

\leavevmode\vadjust pre{\hypertarget{ref-haugenPowerForceVelocityProfilingSprinting2020}{}}%
Haugen TA, Breitschädel F, Samozino P. 2020. Power-{Force}-{Velocity
Profiling} of {Sprinting Athletes}: {Methodological} and {Practical
Considerations When Using Timing Gates}. Journal of Strength and
Conditioning Research. 34(6):1769--1773.
\url{https://doi.org/10.1519/JSC.0000000000002890}

\leavevmode\vadjust pre{\hypertarget{ref-haugenSprintMechanicalVariables2019}{}}%
Haugen TA, Breitschädel F, Seiler S. 2019. Sprint mechanical variables
in elite athletes: {Are} force-velocity profiles sport specific or
individual?Peyré-Tartaruga LA, editor. PLOS ONE. 14(7):e0215551.
\url{https://doi.org/10.1371/journal.pone.0215551}

\leavevmode\vadjust pre{\hypertarget{ref-haugenSprintMechanicalProperties2020}{}}%
Haugen TA, Breitschädel F, Seiler S. 2020. Sprint mechanical properties
in soccer players according to playing standard, position, age and sex.
Journal of Sports Sciences. 38(9):1070--1076.
\url{https://doi.org/10.1080/02640414.2020.1741955}

\leavevmode\vadjust pre{\hypertarget{ref-haugenDifferenceStartImpact2012}{}}%
Haugen TA, Tønnessen E, Seiler SK. 2012. The {Difference Is} in the
{Start}: {Impact} of {Timing} and {Start Procedure} on {Sprint Running
Performance}: Journal of Strength and Conditioning Research.
26(2):473--479. \url{https://doi.org/10.1519/JSC.0b013e318226030b}

\leavevmode\vadjust pre{\hypertarget{ref-haugenSprintRunningPerformance2016}{}}%
Haugen T, Buchheit M. 2016. Sprint {Running Performance Monitoring}:
{Methodological} and {Practical Considerations}. Sports Med
{[}Internet{]}. {[}accessed 2021 Jul 16{]} 46(5):641--656.
\url{https://doi.org/10.1007/s40279-015-0446-0}

\leavevmode\vadjust pre{\hypertarget{ref-jimenez-reyesRelationshipVerticalHorizontal2018}{}}%
Jiménez-Reyes P, Samozino P, García-Ramos A, Cuadrado-Peñafiel V,
Brughelli M, Morin J-B. 2018. Relationship between vertical and
horizontal force-velocity-power profiles in various sports and levels of
practice. PeerJ. 6:e5937. \url{https://doi.org/10.7717/peerj.5937}

\leavevmode\vadjust pre{\hypertarget{ref-jovanovicBmbstatsBootstrapMagnitudebased2020}{}}%
Jovanović M. 2020. Bmbstats: {Bootstrap Magnitude}-based {Statistics}
for {Sports Scientists}. {[}place unknown{]}: Mladen Jovanović, ISBN:
978-8690080359.

\leavevmode\vadjust pre{\hypertarget{ref-R-shorts}{}}%
Jovanović M. 2022. Shorts: Short sprints {[}Internet{]}. {[}place
unknown{]}. \url{https://CRAN.R-project.org/package=shorts}

\leavevmode\vadjust pre{\hypertarget{ref-jovanovic2022}{}}%
Jovanović M, Vescovi J. 2022. {\textbraceleft}Shorts{\textbraceright}:
An r package for modeling short sprints. International Journal of
Strength and Conditioning {[}Internet{]}. 2(1).
\url{https://doi.org/10.47206/ijsc.v2i1.74}

\leavevmode\vadjust pre{\hypertarget{ref-kruschkeDoingBayesianData2015}{}}%
Kruschke JK. 2015. Doing {Bayesian} data analysis: A tutorial with {R},
{JAGS}, and {Stan}. Edition 2. {Boston}: {Academic Press}.

\leavevmode\vadjust pre{\hypertarget{ref-kruschkeRejectingAcceptingParameter2018}{}}%
Kruschke JK. 2018. Rejecting or {Accepting Parameter Values} in
{Bayesian Estimation}. Advances in Methods and Practices in
Psychological Science {[}Internet{]}. {[}accessed 2022 Jul 10{]}
1(2):270--280. \url{https://doi.org/10.1177/2515245918771304}

\leavevmode\vadjust pre{\hypertarget{ref-kruschkeBayesianDataAnalysis2018}{}}%
Kruschke JK, Liddell TM. 2018a. Bayesian data analysis for newcomers.
Psychonomic Bulletin \& Review {[}Internet{]}. {[}accessed 2019 Sep
17{]} 25(1):155--177. \url{https://doi.org/10.3758/s13423-017-1272-1}

\leavevmode\vadjust pre{\hypertarget{ref-kruschkeBayesianNewStatistics2018}{}}%
Kruschke JK, Liddell TM. 2018b. The {Bayesian New Statistics}:
{Hypothesis} testing, estimation, meta-analysis, and power analysis from
a {Bayesian} perspective. Psychonomic Bulletin \& Review {[}Internet{]}.
{[}accessed 2019 Sep 17{]} 25(1):178--206.
\url{https://doi.org/10.3758/s13423-016-1221-4}

\leavevmode\vadjust pre{\hypertarget{ref-makowskiBayestestRDescribingEffects2019}{}}%
Makowski D, Ben-Shachar M, Lüdecke D. 2019. {bayestestR}: {Describing
Effects} and their {Uncertainty}, {Existence} and {Significance} within
the {Bayesian Framework}. Journal of Open Source Software
{[}Internet{]}. {[}accessed 2019 Sep 23{]} 4(40):1541.
\url{https://doi.org/10.21105/joss.01541}

\leavevmode\vadjust pre{\hypertarget{ref-mangineSpeedForcePower2014}{}}%
Mangine GT, Hoffman JR, Gonzalez AM, Wells AJ, Townsend JR, Jajtner AR,
McCormack WP, Robinson EH, Fragala MS, Fukuda DH, Stout JR. 2014. Speed,
{Force}, and {Power Values Produced From Nonmotorized Treadmill Test Are
Related} to {Sprinting Performance}: Journal of Strength and
Conditioning Research. 28(7):1812--1819.
\url{https://doi.org/10.1519/JSC.0000000000000316}

\leavevmode\vadjust pre{\hypertarget{ref-marcote-pequenoAssociationForceVelocity2019}{}}%
Marcote-Pequeño R, García-Ramos A, Cuadrado-Peñafiel V,
González-Hernández JM, Gómez MÁ, Jiménez-Reyes P. 2019. Association
{Between} the {Force}\textendash{{Velocity Profile}} and {Performance
Variables Obtained} in {Jumping} and {Sprinting} in {Elite Female Soccer
Players}. International Journal of Sports Physiology and Performance.
14(2):209--215. \url{https://doi.org/10.1123/ijspp.2018-0233}

\leavevmode\vadjust pre{\hypertarget{ref-morinSpreadsheetSprintAcceleration2017}{}}%
Morin JB. 2017. A spreadsheet for {Sprint} acceleration
{Force}-{Velocity}-{Power} profiling {[}Internet{]}. {[}accessed 2020
Oct 27{]}.
\url{https://jbmorin.net/2017/12/13/a-spreadsheet-for-sprint-acceleration-force-velocity-power-profiling/}

\leavevmode\vadjust pre{\hypertarget{ref-morinSpreadsheetSprintAcceleration2019}{}}%
Morin J-B, Samozino P. 2019. Spreadsheet for {Sprint} acceleration
force-velocity-power profiling.

\leavevmode\vadjust pre{\hypertarget{ref-morinSimpleMethodComputing2019}{}}%
Morin J-B, Samozino P, Murata M, Cross MR, Nagahara R. 2019. A simple
method for computing sprint acceleration kinetics from running velocity
data: {Replication} study with improved design. Journal of Biomechanics.
94:82--87. \url{https://doi.org/10.1016/j.jbiomech.2019.07.020}

\leavevmode\vadjust pre{\hypertarget{ref-motulskyIntuitiveBiostatisticsNonmathematical2018}{}}%
Motulsky H. 2018. Intuitive biostatistics: A nonmathematical guide to
statistical thinking. Fourth edition. {New York}: {Oxford University
Press}.

\leavevmode\vadjust pre{\hypertarget{ref-R-base}{}}%
R Core Team. 2022. R: A language and environment for statistical
computing {[}Internet{]}. Vienna, Austria: R Foundation for Statistical
Computing. \url{https://www.R-project.org/}

\leavevmode\vadjust pre{\hypertarget{ref-samozinoSimpleMethodMeasuring2016}{}}%
Samozino P, Rabita G, Dorel S, Slawinski J, Peyrot N, Saez de Villarreal
E, Morin J-B. 2016. A simple method for measuring power, force, velocity
properties, and mechanical effectiveness in sprint running: {Simple}
method to compute sprint mechanics. Scandinavian Journal of Medicine \&
Science in Sports. 26(6):648--658.
\url{https://doi.org/10.1111/sms.12490}

\leavevmode\vadjust pre{\hypertarget{ref-stenrothSpreadsheetSprintAcceleration2020}{}}%
Stenroth L, Vartiainen P. 2020. Spreadsheet for sprint acceleration
force-velocity-power profiling with optimization to correct start time.
\url{https://doi.org/10.13140/RG.2.2.12841.83045}

\leavevmode\vadjust pre{\hypertarget{ref-stenrothForcevelocityProfilingIce2020}{}}%
Stenroth L, Vartiainen P, Karjalainen PA. 2020. Force-velocity profiling
in ice hockey skating: Reliability and validity of a simple, low-cost
field method. Sports Biomechanics.:1--16.
\url{https://doi.org/10.1080/14763141.2020.1770321}

\leavevmode\vadjust pre{\hypertarget{ref-vescoviSprintMechanicalCharacteristics2021}{}}%
Vescovi JD, Jovanović M. 2021. Sprint {Mechanical Characteristics} of
{Female Soccer Players}: {A Retrospective Pilot Study} to {Examine} a
{Novel Approach} for {Correction} of {Timing Gate Starts}. Front Sports
Act Living {[}Internet{]}. {[}accessed 2021 Jul 1{]} 3:629694.
\url{https://doi.org/10.3389/fspor.2021.629694}

\leavevmode\vadjust pre{\hypertarget{ref-ward-smithEnergyConversionStrategies2001}{}}%
Ward-Smith AJ. 2001. Energy conversion strategies during 100 m
sprinting. Journal of Sports Sciences. 19(9):701--710.
\url{https://doi.org/10.1080/02640410152475838}

\end{CSLReferences}



\end{document}
